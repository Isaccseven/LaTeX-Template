% ~~~~~~~~~~~~~~~~~~~~~~~~~~~~~~~~~~~~~~~~~~~~~~~~~~~~~~~~~~~~~~~~~~~~~~ %
% ~~~~~~~~~~~~~~~~~~~~~~~~~~~~~~ License ~~~~~~~~~~~~~~~~~~~~~~~~~~~~~~~ %
% ~~~~~~~~~~~~~~~~~~~~~~~~~~~~~~~~~~~~~~~~~~~~~~~~~~~~~~~~~~~~~~~~~~~~~~ %

% CC BY-NC-SA 3.0 (http://creativecommons.org/licenses/by-nc-sa/3.0/)
% Siehe auch LICENCE.txt


% ~~~~~~~~~~~~~~~~~~~~~~~~~~~~~~~~~~~~~~~~~~~~~~~~~~~~~~~~~~~~~~~~~~~~~~ %
% ~~~~~~~~~~~~~~~~~~~~~~~~~~~~~~ Präambel ~~~~~~~~~~~~~~~~~~~~~~~~~~~~~~ %
% ~~~~~~~~~~~~~~~~~~~~~~~~~~~~~~~~~~~~~~~~~~~~~~~~~~~~~~~~~~~~~~~~~~~~~~ %

% Artikel-Klasse mit a4paper Option für die Seitenränder
\documentclass[12pt,a4paper]{article}
\usepackage[utf8]{inputenc}

% Zur Anpassung der Seitenränder
\usepackage[a4paper,left=3cm,right=2cm,top=3cm,bottom=2.5cm]{geometry}

% Zeilenabstände anpassen
\usepackage{setspace}

% Deutsch: https://de.overleaf.com/learn/German
\usepackage[T1]{fontenc}
\usepackage[ngerman]{babel}

% Anführungszeichen unten: "`
% Anführungszeichen oben: "'
\usepackage{csquotes}

% um Grafiken einzubinden, {Angabe des Pfades der Bilder}
\usepackage{graphicx}
\graphicspath{ {images/} }

% zur korrekten Platzierung der Banner im Titelblatt
\usepackage{chngpage}
\usepackage{calc}
\usepackage{float}

% Automatisches Generieren von Hyperlinks bei Verweisen und URLs
\usepackage{url}
\usepackage{hyperref}
\hypersetup{
  colorlinks   = true,     % Colours links instead of ugly boxes
  urlcolor     = black,    % Colour for external hyperlinks
  linkcolor    = black,    % Colour of internal links
  citecolor    = black     % Colour of citations
}

% sodass LoT, LoF und Literatur in ToC erscheint
\usepackage[nottoc,numbib]{tocbibind}

% Listingverzeichnis
\usepackage{listings}
\renewcommand{\lstlistlistingname}{Listingverzeichnis}

% LoT und LoF nur ausgeben, wenn es Einträge gibt
% Mit diesem Packet werden alle vorhandenen Grafiken, Tabellen und Listings gezählt
\usepackage[figure,table,lstlisting]{totalcount}

% Akronyme mit Einstellungen bezüglich Gestaltung des Verzeichnisses
\usepackage[acronym, nogroupskip, nonumberlist, nopostdot]{glossaries}
\loadglsentries{environment/acronym}
\makenoidxglossaries
\setacronymstyle{long-sc-short}

% Einstellen von environments und captions
\usepackage[hang]{footmisc}
\usepackage{wrapfig}
\usepackage[font=small, justification=centering]{caption}
\usepackage{booktabs}
\usepackage{enumitem}
\setlist[itemize]{itemsep=0cm}

% Setzen von Abständen innerhalt von Fußnoten: für mehr Möglichkeiten siehe https://latex.org/forum/viewtopic.php?t=6781
\renewcommand{\footnotemargin}{12pt}

% Biblatex als Literaturverwaltung
% Siehe auch die Datei biblatex.cfg
% Dokumentation: https://ctan.kako-dev.de/macros/latex/contrib/biblatex/doc/biblatex.pdf
\usepackage[style=ext-authoryear,     % ext- ermöglicht das Einblenden der Klammern um die Jahreszahl in Fußnoten
            sorting=nyt,              % Nach Nachnamen des ersten genannten Autorens sortieren, dann Jahr, dann Titel
            isbn=false,               % Ausblenden des Feldes
            url=false,                % Ausblenden des Feldes
            doi=true,                 % Einblenden des Feldes
            eprint=false,             % Ausblenden des Feldes
            maxcitenames=2,           % Ab drei Autoren mit "et at." abkürzen
            maxbibnames=100]{biblatex}% Alle Autoren im Literaturverzeichnis ausschreiben

% Einbinden der Literatureinträge
\addbibresource{references.bib}
\addbibresource{zotero-references.bib}

% Zeilenumbrüche in Url bei jedem beliebigen Buchstaben,
% um overfull H-Boxes im Literaturverzeichnis zu vermeiden
\setcounter{biburllcpenalty}{1000}


% ~~~~~~~~~~~~~~~~~~~~~~~~~~~~~~~~~~~~~~~~~~~~~~~~~~~~~~~~~~~~~~~~~~~~~~ %
% ~~~~~~~~~~~~~~~~~~~~~~~~~~~ Konfiguration ~~~~~~~~~~~~~~~~~~~~~~~~~~~~ %
% ~~~~~~~~~~~~~~~~~~~~~~~~~~~~~~~~~~~~~~~~~~~~~~~~~~~~~~~~~~~~~~~~~~~~~~ %

% Projektspezifische Einstellungen laden
% In dieser Datei müssen lediglich die vorgegebenen Variablen geändert werden
% ~~~~~~~~~~~~~~~~~~~~~~~~~~~~~~~~~~~~~~~~~~~~~~~~~~~~~~~~~~~~~~~~~~~~~~ %
% ~~~~~~~~~~~~~~~~~~~~~~ Erweiterung des Präambel ~~~~~~~~~~~~~~~~~~~~~~ %
% ~~~~~~~~~~~~~~~~~~~~~~~~~~~~~~~~~~~~~~~~~~~~~~~~~~~~~~~~~~~~~~~~~~~~~~ %

% Shortcuts für häufig wiederkehrende Begriffe
\newcommand{\asw}{ASW -- Berufsakademie Saarland}

% Wenn nach den 3 Ebenen nach \subsubsection eine weitere Gliederungsebene benötigt wird
\newcommand{\paragraphheader}[1]{\paragraph{#1}\mbox{}\\}

% Vordefinierte Arten der Fußnoten. 
% Die Struktur kann an dieser Stelle geändert werden und betrifft jede Verwendung im gesamten Dokument,
% was praktisch sein kann, falls der Betreuer oder Gutachter zb. die ausgeschriebene Variante von "Vgl." bevorzugt.
\newcommand{\vgl}[2]{\footcite[Vgl.][#2]{#1}}
\newcommand{\footrefnote}[2]{\footnote{Für das Thema #1 siehe Kapitel~\ref{#2}}}
\newcommand{\wholesection}[2]{\footcite[Für den gesamten Abschnitt vgl.][#2]{#1}}

% Zur Generierung von Blindtext - kann entfernt werden
\usepackage{lipsum}

% ~~~~~~~~~~~~~~~~~~~~~~~~~~~~~~~~~~~~~~~~~~~~~~~~~~~~~~~~~~~~~~~~~~~~~~ %
% ~~~~~~~~~~~~~~~~~~~ Informationen über die Arbeit ~~~~~~~~~~~~~~~~~~~~ %
% ~~~~~~~~~~~~~~~~~~~~~~~~~~~~~~~~~~~~~~~~~~~~~~~~~~~~~~~~~~~~~~~~~~~~~~ %

\title{Entwicklung einer REST API mit Hilfe des Spring Frameworks}
\author{Luca Henn}
\date{}

% Weitere Variablen innerhalb von LaTeX für das Titelblatt
\newcommand{\varMartrikelnummer}{IF022003}
\newcommand{\varArbeit}{Studienarbeit}
\newcommand{\varStudiengang}{Wirtschaftsinformatik}
\newcommand{\varUnternehmen}{Deutsche Telekom AG}
\newcommand{\varBetrBetreuer}{Patrick Luckas}
\newcommand{\varASWGutachter}{Prof. Dr. Dieter Hofbauer}
\newcommand{\varEingereichtAm}{01. Januar 2021}


% ~~~~~~~~~~~~~~~~~~~~~~~~~~~~~~~~~~~~~~~~~~~~~~~~~~~~~~~~~~~~~~~~~~~~~~ %
% ~~~~~~~~~~~~~~~~~~~~~~~~~~~~ Customizing ~~~~~~~~~~~~~~~~~~~~~~~~~~~~~ %
% ~~~~~~~~~~~~~~~~~~~~~~~~~~~~~~~~~~~~~~~~~~~~~~~~~~~~~~~~~~~~~~~~~~~~~~ %

% Titelblatt: environment/titlepage | environment/titlepage_modern
\newcommand{\varTitlepage}{environment/titlepage_modern}

% Der Name der Datei für das Firmenlogo. Falls keins vorhanden ist wird es aber keinen Fehler geben.
% Die Datei muss im Verzeichnis der Bilder liegen und wird ohne Endung hier angegeben.
\newcommand{\varCompanyLogoFile}{logo_company}

% Wenn environment/titlepage gewählt und \varCompanyLogoFile gesetzt ist
% muss folgende Angabe an das Logo angepasst werden um die Bilder in die vertikale Mitte des Banners zu setzen
% Der passende Wert kann von 25pt abweichen (am besten per trial-and-error herausfinden)
%\newcommand{\varTitlepageLogoMarginTop}{25pt}

% Schriftart: Times New Roman
\renewcommand{\rmdefault}{ptm}

% Ob in den Verzeichnissen eine deklarative Abkürzung stehen soll (zB. "Abb. 1")
% Wenn es nicht gewollt ist muss die nächste Zeile auskommentiert werden.
\newcommand{\varShowTitlesInLists}{true}

% Klammern um die Jahresangabe in Fußnoten und im Literaturverzeichnis ausblenden
% Wenn es nicht gewollt ist muss die nächste Zeile auskommentiert werden.
%\renewcommand{\varNoParenthesesAroundYear}{true}

% Ob noch vor dem Inhaltsverzeichnis ein Sperrvermerk angezeigt werden soll.
% Der Sperrvermerk kann in environment/sperrvermerk.tex angepasst werden.
% Wenn es nicht gewollt ist muss die nächste Zeile auskommentiert werden.
\newcommand{\varShowBlockingNote}{true}


% ~~~~~~~~~~~~~~~~~~~~~~~~~~~~~~~~~~~~~~~~~~~~~~~~~~~~~~~~~~~~~~~~~~~~~~ %
% ~~~~~~~~~~~~~~~~~~~~~~~~~~~ Silbentrennung ~~~~~~~~~~~~~~~~~~~~~~~~~~~ %
% ~~~~~~~~~~~~~~~~~~~~~~~~~~~~~~~~~~~~~~~~~~~~~~~~~~~~~~~~~~~~~~~~~~~~~~ %

% Falls LaTeX Wörter nicht richtig trennt können hier eigene Angaben in folgendem Stil gemacht werden
\hyphenation{Java-Script}

% Ob "Abb.", "Tbl." und "Lst." vor den Nummern der Verzeichnisse erscheinen
\ifdefined\varShowTitlesInLists
  \makeatletter
  \renewcommand{\l@figure}[2]{\@dottedtocline{1}{1.5em}{2.3em}{Abb. #1}{#2}}
  \renewcommand{\l@table}[2]{\@dottedtocline{1}{1.5em}{2.3em}{Tbl. #1}{#2}}
  \renewcommand{\l@lstlisting}[2]{\@dottedtocline{1}{1.5em}{2.3em}{Lst. #1}{#2}}
  \makeatother
\fi

% Ausblenden der Klammern um die Jahresangabe im Literaturverzeichnis
\ifdefined\varNoParenthesesAroundYear
  \makeatletter
  \def\act@on@bibmacro#1#2{%
    \expandafter#1\csname abx@macro@\detokenize{#2}\endcsname
  }
  \def\patchbibmacro{\act@on@bibmacro\patchcmd}
  \def\pretobibmacro{\act@on@bibmacro\pretocmd}
  \def\apptobibmacro{\act@on@bibmacro\apptocmd}
  \def\showbibmacro{\act@on@bibmacro\show}
  \makeatother

  \patchbibmacro{date+extradate}{%
  \printtext[parens]%
  }{%
  \setunit{\addperiod\space}%
  \printtext%
  }{}{}
\fi


% ~~~~~~~~~~~~~~~~~~~~~~~~~~~~~~~~~~~~~~~~~~~~~~~~~~~~~~~~~~~~~~~~~~~~~~ %
% ~~~~~~~~~~~~~~~~~~~~~~~~~~~~~~ Dokument ~~~~~~~~~~~~~~~~~~~~~~~~~~~~~~ %
% ~~~~~~~~~~~~~~~~~~~~~~~~~~~~~~~~~~~~~~~~~~~~~~~~~~~~~~~~~~~~~~~~~~~~~~ %
\setcounter{tocdepth}{4}
\setcounter{secnumdepth}{4}

\begin{document}

  % Titelblatt
  \input{\varTitlepage}
  \newpage

  \ifdefined\varShowBlockingNote
    \setstretch{1.2} 
    \section*{Sperrvermerk}
Die vorliegende \varArbeit \space basiert auf internen, vertraulichen Daten und Informationen der \varUnternehmen. 
In diese Arbeit dürfen Dritte, mit Ausnahme der Gutachter und befugten Mitgliedern des Prüfungsausschusses, 
ohne ausdrückliche Zustimmung des Unternehmens und des Verfassers keine Einsicht nehmen. Eine Vervielfältigung 
und Veröffentlichung der \varArbeit \space ohne ausdrückliche Genehmigung – auch auszugsweise – ist nicht erlaubt.
    \setstretch{1} 
    \pagebreak
  \fi

  % Seitenzahlen auf große römische Zahlen umstellen
  \pagenumbering{Roman}

  % Inhaltsverzeichnis
  \tableofcontents
  \newpage
  
  % Abbildungsverzeichnis
  \iftotalfigures
    \listoffigures
  \fi

  % Tabellenverzeichnis
  \iftotaltables
    \listoftables
  \fi

  % Listingverzeichnis
  \iftotallstlistings
    \addcontentsline{toc}{section}{Listingverzeichnis}
    \lstlistoflistings
  \fi
  \newpage

  % Abkürzungsverzeichnis
  \addcontentsline{toc}{section}{Abkürzungsverzeichnis}
  \setstretch{0.5} 
  \printnoidxglossary[type=acronym,sort=letter,style=listdotted,title=Abkürzungsverzeichnis]
  \newpage

  % Arabische Seitennummerierung für den Hauptteil
  \pagenumbering{arabic}

  % Um den Zeilenabstand der Wordvorlage anzupassen
  \setstretch{1.2}
  
  % Inhalt der Arbeit -> Strukturierung in section/root
  \section{Einleitung}
  \label{Einleitung}  
  In der heutigen Zeit gewinnen Mircoservices immer mehr an Beliebtheit. Dies liegt nicht nur an der vergleichbar besseren Wartbarkeit gegenüber den Monolithen, sondern auch an der schieren Größe heutiger IT-Systeme. Der Informationstransport kann hierbei jedoch nicht über den direkten Weg, wie bei Monolithen stattfinden, sondern muss über extra entwickelte Programmierschnittstellen ablaufen. In der Regel werden hierfür meist REST-Schnittstellen verwendet. Ein weit verbreitetes Framework im Java-Enterprise Bereich ist Spring, oder vielmehr Spring-Boot. Spring-Boot vereinfacht die Entwicklung von REST-Schnittstellen und ermöglicht somit die Entwicklung von wartbaren Mircoservices im Enterprise-Bereich. 
\newline
\newline
Das erste Kapitel, die Grundlegenden Webtechnologien, führt in die Thematik der Web-Entwicklung ein. Hierbei wird mit der Basistechnologie HTTP begonnen und auf dessen Teilthemen, HTTP-Methoden und HTTP-Statuscodes eingegangen. Die folgende grundlegende Webtechnologie auf die eingegangen wird, ist das Softwareparadigma REST und dessen Zustandslosigkeit. Zu Letzt wird auf die grundlegende Funktionsweise und die Anwendungszwecke von API eingegangen. 
\pagebreak

\section{Grundlegende Webtechnologien}
  \label{Einführung}
  \subsection{Erklärung von HTTP}
  \label{Erklärung von HTTP}
  \subsubsection{Was ist HTTP}
    \label{Was ist HTTP }
    \input{sections/grundlegende_webtechnologien/erklärung_von_http/was_ist_http}

\subsubsection{Historie von HTTP}
    \label{Historie von HTTP}
    \input{sections/grundlegende_webtechnologien/erklärung_von_http/historie_von_http}
\pagebreak

\subsubsection{Methoden}
  \label{HTTP-Methoden}
  \input{sections/grundlegende_webtechnologien/erklärung_von_http/methoden}
\pagebreak

\subsubsection{Statuscodes}
  \label{HTTP-Statuscodes}
  \input{sections/grundlegende_webtechnologien/erklärung_von_http/statuscodes}
\pagebreak
\pagebreak

\subsection{Erklärung von REST}
  \label{Erklärung von REST}
  \subsection{Was ist REST ?}
\label{Was ist REST ?}
\include{sections/grundlegende_webtechnologien/erklärung_von_rest/was_ist_rest}


\subsubsection{Prinzipien}
  \label{Prinzipien}
\pagebreak

\paragraph{Client-Server}
\label{Client-Server}
\include{sections/grundlegende_webtechnologien/erklärung_von_rest/prinzipien/client_server}
\pagebreak

\paragraph{Zustandslosigkeit}
\label{Zustandslosigkeit}
\include{sections/grundlegende_webtechnologien/erklärung_von_rest/prinzipien/zustandslosigkeit}
\pagebreak


\paragraph{Caching}
\label{Caching}
\include{sections/grundlegende_webtechnologien/erklärung_von_rest/prinzipien/caching}
\pagebreak

\paragraph{Einheitliche Schnittstelle}
\label{Einheitliche Schnittstelle}
\include{sections/grundlegende_webtechnologien/erklärung_von_rest/prinzipien/einheitliche_schnittstelle}
\pagebreak

\paragraph{Mehrschichtige Systeme}
\label{Mehrschichtige Systeme}
\include{sections/grundlegende_webtechnologien/erklärung_von_rest/prinzipien/mehrschichtige_systeme}
\pagebreak

\paragraph{Code-on-Demand (optional)}
\label{Code-on-Demand}
\include{sections/grundlegende_webtechnologien/erklärung_von_rest/prinzipien/code_on_demand}
\pagebreak


\pagebreak

\subsection{Einführung in APIs}
  \label{Einführung in APIs}
  \subsubsection{Was ist eine API}
  \label{Was ist eine API}
  \input{sections/grundlegende_webtechnologien/einführung_in_apis/was_ist_eine_api}
\pagebreak

\subsubsection{Anwendungszwecke}
  \label{Anwendungszwecke}
  \input{sections/grundlegende_webtechnologien/einführung_in_apis/anwendungszwecke}
\pagebreak
\pagebreak
\pagebreak

\section{Entwicklung einer REST API mit Hilfe des Spring Frameworks}
  \label{Entwicklung einer REST API mit Hilfe des Spring Frameworks}
  \subsection{Einführung in das Spring Framework}
    \label{Einführung in das Spring Framework}
    Das Spring Framework wurde 2003 entwickelt und ist ein Open Source Framework für die Entwicklung moderner java-basierten enterprise Anwendungen. Es reduziert dabei die „ ... Komplexität der standardmäßigen Java-Spezifikation („J2EE“ bzw. „Java Plattform“) sowie des Komponentenmodells Enterprise JavaBeans („EJB“) bedeutend ... "\footcite{springbootionos}.
\pagebreak

\subsection{Erstellung eines Spring Projekts}
  \label{Erstellung eines Spring Projekts}
  \input{sections/entwicklung_einer_rest_api_mit_hilfe_des_spring_frameworks/erstellung_eines_spring_projects}
\pagebreak

\subsection{REST Architektur unter Spring}
  \label{REST Architektur unter Spring}
  \input{sections/entwicklung_einer_rest_api_mit_hilfe_des_spring_frameworks/rest_architektur_unter_spring}
\pagebreak

\subsection{Verarbeitung von Requests}
  \label{Verarbeitung von Requests}
  \input{sections/entwicklung_einer_rest_api_mit_hilfe_des_spring_frameworks/verarbeitung_von_requests}
\pagebreak
\pagebreak

\appendix
\section{Listings}
\label{appendix_dummy}
  Hier ein Beispiel für einen Anhang.
  Gleichzeitig die Demo für ein Listing, indem es die Quelldatei für den Anhang hier einfügt.
  Allerdings sollten Grundeinstellungen, wie in der README vorgenommen werden, wenn Listings verwendet werden
  \lstinputlisting[label=lst_lipsum,caption=lipsum.tex,language=tex]{sections/lipsum.tex}
  \newpage

  % Literaturverzeichnis
  \setstretch{1.1} 
  \sloppy
  \hbadness=2000
  \printbibliography[heading=bibintoc]
  \newpage

  % Persönliche Erklärung
  \pagenumbering{gobble}
  \section*{Persönliche Erklärung}

Hiermit erkläre ich, dass ich
\begin{enumerate}
  \item meine \varArbeit \space ohne fremde Hilfe angefertigt habe,
  \item die Übernahme wörtlicher Zitate aus der Literatur sowie die Verwendung von Gedanken anderer Autoren an den entsprechenden Stellen innerhalb der Arbeit gekennzeichnet habe und
  \item meine \varArbeit \space bei keiner anderen Prüfungsstelle vorgelegt habe.
\end{enumerate}
Ich bin mir bewusst, dass eine falsche Erklärung zum Nichtbestehen der \varArbeit \space führt.

\vspace{2cm}

\begin{tabular}{lp{2em}l}
 \hspace{5cm}   && \hspace{3cm} \\\cline{1-1}\cline{3-3}
 Ort, Datum     && Unterschrift
\end{tabular}


\end{document}