In der heutigen Zeit gewinnen Mircoservices immer mehr an Beliebtheit. Dies liegt nicht nur an der vergleichbar besseren Wartbarkeit gegenüber den Monolithen, sondern auch an der schieren Größe heutiger IT-Systeme. Der Informationstransport kann hierbei jedoch nicht über den direkten Weg, wie bei Monolithen stattfinden, sondern muss über extra entwickelte Programmierschnittstellen ablaufen. In der Regel werden hierfür meist REST-Schnittstellen verwendet. Ein weit verbreitetes Framework im Java-Enterprise Bereich ist Spring, oder vielmehr Spring-Boot. Spring-Boot vereinfacht die Entwicklung von REST-Schnittstellen und ermöglicht somit die Entwicklung von wartbaren Mircoservices im Enterprise-Bereich. 
\newline
\newline
Das erste Kapitel, die Grundlegenden Webtechnologien, führt in die Thematik der Web-Entwicklung ein. Hierbei wird mit der Basistechnologie HTTP begonnen und auf dessen Teilthemen, HTTP-Methoden und HTTP-Statuscodes eingegangen. Die folgende grundlegende Webtechnologie auf die eingegangen wird, ist das Softwareparadigma REST und dessen Zustandslosigkeit. Zu Letzt wird auf die grundlegende Funktionsweise und die Anwendungszwecke von API eingegangen. 