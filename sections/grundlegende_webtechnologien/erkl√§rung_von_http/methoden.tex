Die HTTP Methoden beschreiben die verschiedenen Anfragemöglichkeiten, mit denen auf eine Ressource zugegriffen werden kann. Die möglichen Methoden die dabei zur Verfügung stehen, lauten GET, HEAD, POST, PUT, DELETE, CONNECT, OPTIONS, TRACE und PATCH. \newline
Die GET-Methode wird dazu genutzt, eine Ressource von einem Server abzufragen. Dabei sollte die GET-Methode nur für das Zurückliefern von Daten verwendet werden. \newline
Die HEAD-Methode liefert prinzipiell das Gleiche wie die GET-Methode zurück, jedoch ohne den, im Request mitgeschickten, Request-Body. \newline
Die POST Methode liefert, nicht wie bei den vorherigen Methoden, die Sachen vom Server zurück zum Client, sondern empfängt eine, vom Client ausgehende, Ressource. \newline
Die Methode funktioniert grundsätzlich ähnlich wie die POST-Methode, da Sie ebenfalls Ressourcen bzw. genauer Entities erstellen kann. Die Unterschied liegt jedoch darin, das PUT zusätzlich zum Erstellen, auch noch eine Entity updaten kann. Das bedeutet, dass bei bereits vorhanden sein einer Entity, diese, mit den im Request-Body mitgelieferten Daten, aktualisiert wird. \newline
Die darauf folgende Methode lautet DELETE und wird, wie der Name schon sagt, benutzt, wenn eine Ressource gelöscht werden soll.