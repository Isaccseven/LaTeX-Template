HTTP steht für Hypertext Transfer Protocol und ist eine der drei Basistechnologien, zusammen mit Uniform Resource Locators und der Hypertext Markup Language, auf denen Tim Berners-Lee das grundlegende Web entwickelt hat. \vgl{pollard_http2_2019}{S.4--9}
Es ist ein application-layer protocol das für das Versenden von hypermedia documents, wie HTML entwickelt wurde. HTTP wurde für die Kommunikation zwischen dem Client, beispielsweise einem Webbrowser und einem Server designt. Die Entwicklung basiert auf dem klassischen Client-Server-Model, wobei ein Client eine Anfrage an den Server schickt und wartet, bis er eine Antwort zurückbekommt. Des Weiteren ist HTTP ein zustandsloses Protokoll. Dies bedeutet, dass keinerlei Daten auf dem Server zwischen unterschiedlichen Anfragen gespeichert werden.
\vgl{mdn_contributors_http_2021}{}
