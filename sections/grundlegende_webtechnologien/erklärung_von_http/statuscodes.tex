Das HTTP-Protokoll liefert bei der Ankunft einer Response einen Statuscode. Diese Statuscodes werden in die Gruppe 1xx, 2xx, 3xx, 4xx, 5xx. Die 100er Gruppe liefert hierbei eine rein informelle Rückmeldung und werden als Informational responses betitelt. Die häufigste Vertreter dieser Gruppierung ist der 100 Continue, der dem Client zurückgibt, dass dieser mit dem Request entweder fortfahren, oder diesen bei bereits fertiger Ausführung, ignorieren soll.    \newline
Die 200er Gruppe sind die Successful responses. Diese können bei den HTTP-Methoden GET, im Falle einer erfolgreich zurückgelieferten Repräsentation einer Ressource, bei HEAD, wenn nur der header und nicht der message body erfolgreich zurückgeliefert wurde, bei PUT oder POST, wenn die Ressource erfolgreich an einen Server übertragen werden konnte oder TRACE, wenn der message body in einem request enthalten ist, zurückgegeben werden. Die bekanntesten Vertreter dieser Gruppierung sind der 200 OK und der 201 Created. Der 200 OK kann bei den HTTP-Methoden GET, HEAD, PUT/POST und TRACE vorkommen und wird zurückgegeben, wenn eine, für die jeweilige Methode, erfolgreiche Response zurückgeliefert wird. Der 201 Created wird typischerweise zurückgeliefert, wenn bei einem POST oder PUT request erfolgreich eine Ressource erstellt werden konnte.  \newline
Die dritte Gruppe ist die Gruppe der 300er Statusmeldung. Sie sind die Redirection messages, die Zurückgeben, ob sich die URL der Ressource beispielsweise permanent oder nur temporär verändert hat. \newline
Die vierte Gruppe ist die Gruppe der Client error responses und bezieht sich auf die Fehler, die vom Client ausgehen. Die responses, die in die Gruppe am öftesten vertreten sind, sind 400 Bad Request. Er gibt zurück, dass der Server den Request aufgrund falscher Syntax sind verstanden hat. Ein weiterer sehr bekannter Statuscode ist der 401 Unauthorized. Dieser bezieht sich darauf, dass der Client nicht autorisiert ist und sich erst autorisieren muss, um die Ressource abfragen zu dürfen. Ähnlich ist auch der 403 Forbidden, der aber, im Vergleich zum 401 Unauthorized, die Identität des Client bereits kennt und und weiß, dass dieser nicht die Rechte hat, diese Ressource abfragen zu dürfen. Der bekannteste Statuscode ist jedoch vermutlich der 404 Not Found, da er angibt, dass eine Ressource nicht zu finden ist. Er ist oft zu sehen auf der Fehlerseite von Webseiten. \newline
Die letzte Gruppe ist die Gruppe der 500er, die Server error responses. Diese geben, dem Namen entsprechend, serverseitige Fehler zurück. Die bekanntesten Statuscodes sind hierbei 500 Internal Server Error, 501 Not Implemented, 502 Bad Gateway und 503 Service Unavailable. Der Statuscode 500 wird zurückgeliefert, wenn der Server nicht weiß, wie er den Request verarbeiten soll. Dies geschieht beispielsweise, wenn statt einer Zahl, ein Zeichen in einer Form verwendet wird. Der 501 Not Implemented error wird dann zurückgeliefert, wenn die verwendete HTTP-Methode vom Server nicht unterstützt und somit auch nicht verwendet werden kann. Der Statuscode 502 Bad Gateway gibt wiederum zurück, dass der Server eine ungültige Response vom einem Gateway zurück bekommen hat. Zu Letzt der 503 Service Unavailable gibt, wie der Name bereits vermuten lässt, das nicht Erreichen des Servers zurück. Dies kann z.B. an einem Wartungsupdate und einem damit zusammenhängendem Neustart zutun haben.